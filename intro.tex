%%%%%%%%%%%%%%%%%%%%%%%%%%%%%%%%%%%%%%%%%%%%%%%%%%%%%%%%%%%%%%%%%%%%%%%%%%%%%%%
% intro.tex: Introduction to the thesis
%%%%%%%%%%%%%%%%%%%%%%%%%%%%%%%%%%%%%%%%%%%%%%%%%%%%%%%%%%%%%%%%%%%%%%%%%%%%%%%%
\chapter{Introduction}
\label{intro_chapter}
%%%%%%%%%%%%%%%%%%%%%%%%%%%%%%%%%%%%%%%%%%%%%%%%%%%%%%%%%%%%%%%%%%%%%%%%%%%%%%%%

Each of the four fundamental forces of the universe is described by a particular theory. 
Gravity is understood through the techniques of general relativity, while electromagnetism and the weak nuclear force are united in the electroweak theory of the standard model.
Quantum chromodynamics (QCD) describes the behavior of the strong nuclear force, which is the subject of this thesis.

The strong nuclear force binds together the constituents of nuclear matter.
The fundamental constituents of the theory are quarks and the force-carrying particles known as gluons.
The strength of an interaction is characterized by a parameter known as the coupling constant. 
Traditional perturbation techniques involve a series expansion in this constant, which works well when the coupling constant is small.
When the coupling constant is large, each successive term in this expansion is larger than the preceding one, so the perturbation expansion is not useful.
The value of the coupling constant of QCD varies with energy scale.
At high energies, the coupling constant is small, but the strength of the interaction grows at lower energies.
At these energies, quarks and gluons are confined within particles known as hadrons, and investigating their dynamics requires new techniques.

Beginning in the late 1990's, new techniques were proposed to tackle these non-perturbative problems.
In the framework of string theory it was conjectured that a duality exists between strongly-coupled gauge theories and weakly-coupled gravitational theories.
Calculations that are analytically intractable in the field theory can be related to more easily calculated results from the gravity theory. 
These models are known as gauge/gravity dualities.
One proposed gauge/gravity model is the AdS/CFT (Anti-de Sitter space/Conformal Field Theory) correspondence, which relates certain strongly-coupled field theories to weakly-coupled gravitational theories with an extra dimension.
The potential application of the AdS/CFT correspondence to non-perturbative aspects of QCD was quickly recognized, although there are some assumptions in the correspondence that do not apply to QCD.
While a gravitational dual to QCD has not been discovered, there has been much success over the past fifteen years in developing five-dimensional models that capture key features of hadron phenomenology.
These effective models are known as AdS/QCD.

This thesis is organized as follows
\begin{itemize}

\item In Chapter \ref{sec:Soft-Wall-Model}, we describe previous work done on soft-wall AdS/QCD models. 
We cover the application of the model to the calculation of meson spectra and the modeling of chiral symmetry breaking.

\item In Chapter 3, we detail the construction of a three-field dynamical AdS/QCD model.
The need for a third background field is motivated by the analysis of chiral symmetry breaking. 
We derive the equations of motion for the background fields, and construct a potential with the correct behavior.
We describe the numerical techniques needed to derive the form of the background fields from this potential.

\item In Chapter 4, we present the meson spectra that result from the dynamical AdS/QCD model.
We describe the method for calculating the meson spectra and compare the model to experimental results. 


\end{itemize}
%%%%%%%%%%%%%%%%%%%%%%%%%%%%%%%%%%%%%%%%%%%%%%%%%%%%%%%%%%%%%%%%%%%%%%%%%%%%%%%%
