%%%%%%%%%%%%%%%%%%%%%%%%%%%%%%%%%%%%%%%%%%%%%%%%%%%%%%%%%%%%%%%%%%%%%%%%%%%%%%%%
% conclusion.tex:
%%%%%%%%%%%%%%%%%%%%%%%%%%%%%%%%%%%%%%%%%%%%%%%%%%%%%%%%%%%%%%%%%%%%%%%%%%%%%%%%
\chapter{Conclusion and Discussion}
\label{conclusion_chapter}
%%%%%%%%%%%%%%%%%%%%%%%%%%%%%%%%%%%%%%%%%%%%%%%%%%%%%%%%%%%%%%%%%%%%%%%%%%%%%%%%

%%%%%%%%%%%%%%%%%%%%%%%%%%%%%%%%%%%%%%%%%%%%%%%%%%%%%%%%%%%%%%%%%%%%%%%%%%%%%%%%
\begin{flushright}
I'll go higher still!\\
I'll build my throne higher! I can and I will!\\
I'll call some more turtles.  I'll stack 'em to heaven!\\
I need 'bout five thousand, six hundred and seven!\\
Dr. Seuss, ``Yertle the Turtle"
\end{flushright}

This thesis began with an overview of gauge/gravity dualities and a motivational derivation of the anti-de Sitter space/conformal field theory correspondence.
We sketched the string theoretical underpinnings for this duality and its potential usefulness for analyzing strongly-coupled regimes of gauge theories.
With this in mind, we outlined the various applications to the study of quantum chromodynamics, specifically in the realm of hadronic physics, known as AdS/QCD.

In Chapter \ref{sec:Soft-Wall-Model}, we described the framework of a particular AdS/QCD model, known as the soft-wall model. 
This model uses a dilaton field to break the conformal symmetry of the five-dimensional AdS gravitational theory, in contrast to the simple cut-off of the conformal dimension used in the hard-wall model.
Using the simplest form for the dilaton, we showed how to calculate the mass spectrum of the radially excited states of the light mesons.
We showed the improvements that this simple soft-wall model exemplifies in comparison to the hard-wall model, as well as its shortcomings in modeling the ground states of the meson spectrum and in correctly describing the explicit and spontaneous chiral symmetry breaking of QCD.
We then introduced a modified soft-wall model with an improved description of chiral symmetry breaking.
We showed the improved spectra for the scalar, vector, and axial-vector mesons, and explored the unique dynamics of the pseudoscalar sector.
The phenomenological results of this model are quite good, but they still depend on an \emph{ad hoc} parameterization of the dilaton and chiral condensate fields.

In Chapter \ref{ch:dynamical}, we introduced dynamical models of AdS/QCD, where all fields in the model are derived from a potential.
The simplest models consist of a gravity-dilaton action, along with the matter fields that describe the meson content. 
We derived the equations of motion that result from examining the Einstein equation and varying the action with respect to the dilaton field in this model. 
We did not explore the phenomenology of this model, because it lacks a mechanism for chiral symmetry breaking.
Next we described models with a gravity-dilaton-tachyon action, with the closed-string tachyon inserted to break chiral symmetry.
We showed that the equations of motion for these models require a tachyon profile that cannot accurately model the chiral symmetry breaking, as evidenced by the inaccurate axial-vector mass splitting.

Finally, we summarized the results of a dynamical model that keeps the meson fields in the open-string sector, while introducing the gravity-dilaton action in the open-string sector. 
The background dynamics of the model are analyzed by examining the equations of motion for the background fields -- the dilaton and the vacuum expectation value of the scalar meson field.
To obtain the correct phenomenological results for the meson spectra, the chiral condensate field must have different behavior than the modified soft-wall model introduced in Chapter \ref{sec:Soft-Wall-Model}.
Namely, the chiral condensate field becomes asymptotically constant in the infrared limit of the theory, which necessitates a warping of the metric in this regime.
This model obtains good phenomenological results for the meson mass spectra, although it still makes use of a parameterization of the background fields, and the authors do not explicitly determine the scalar potential.
We conclude this chapter by noting that it is impossible to obtain the correct mass spectra for all of the mesons within a model containing only the dilaton and chiral condensate background fields and maintaining a purely AdS metric.

In Chapter \ref{ch:dynamical_threefield}, we described in detail a dynamical AdS/QCD model including three background fields -- the glueball condensate field is included along with the dilaton and chiral condensate.
In order to keep the string frame metric purely AdS, we chose to place all of the fields in the bulk, or closed-string sector, of the theory, in contrast to some models that place the matter fields in the open-string sector.
From a practical standpoint, this modification keeps the overall dilaton factor the same in the string frame action, and simplifies the background equations of motion.
This choice is in keeping with the phenomenological spirit of the original AdS/QCD models.
Absent an action that is shown to be embeddable in string theory, there is no \emph{a priori} reason to prefer one form of the action over the other.
Instead, we make this choice based upon phenomenology and a desire to preserve the AdS metric.

After deriving the background equations of motion for this three-field model, we make an \emph{ansatz} for the potential by examining the asymptotic behavior of the background fields.
The infrared behavior of the background fields is determined by the confining behavior of the meson spectra, namely, the linear trajectories of the higher radial excitations and the constant mass-splitting between the higher radial excitations of the axial-vector and vector mesons.
The ultraviolet asymptotics for the chiral and glueball condensates are set by the AdS/CFT dictionary (\ref{equstandardform}) in relation to their corresponding field theory operator.
We have set the UV behavior for the dilaton field by using the background equation that does not depend on the potential (\ref{C}), rather than by identifying the dilaton with a particular field theory operator. 
By examining the behavior of the equations of motion in the UV and IR limits, we construct an \emph{ansatz} for the potential that includes all allowed combinations of the background fields.

We then parameterize solutions for the background fields that result from the equations of motion including the full potential.
Physical parameters are fit to the meson spectra, while the other free parameters are solved through a numerical minimization procedure that minimizes the error in the background equations.
Our construction allows for an additional term that is a function of the dilaton only, which we find numerically and then fit as a function of the dilaton field.
This work results in a potential that, while not guaranteed to be unique, does allow for solutions to the background field with the correct asymptotic behavior.

This model also yields good phenomenological results for the meson spectra, as shown in Chapter \ref{ch:meson_spectra}.
The $n\ge3$ excitations of the $\rho$ and $a_1$ spectra were used to fit input parameters, as was the location of the ``bend'' of the $a_1$ spectrum.
The fit to all of the experimental data is quite good, as shown in Figures \ref{figRho} and \ref{fig:axial3}.
No additional parameters were used to fit the pion spectrum, which also showed good agreement, as seen in Figure \ref{figPion}.
The ground state pion is massless, which is expected in a model where the quark mass is zero, as discussed in Section \ref{sub:GMOR}.
Within this model, the equation of motion for the scalar $f_0$ mesons mixes with the equation of motion for the scalar glueballs, as explored in Chapter \ref{ch:scalar}.
The mass spectra for the $f_0$ mesons should be calculated and compared to experiment, while the scalar glueball spectrum can be compared to results from lattice QCD and other AdS/QCD models.
An obvious improvement to this model would be to include a nonzero quark mass, which would allow for the calculation of the physical pion mass, and may alter other phenomenological results.
A nonzero quark mass would change the UV asymptotics of the chiral condensate by adding a linear term, which would then affect the derivation of the potential.

This model can be extended to study QCD at finite temperature. 
It has been established experimentally that QCD matter changes phase at a critical temperature $T_c$, becoming a strongly-interacting plasma of freely associated quarks and gluons, rather than the bound states of hadrons \cite{qgp,qgp3,phases}.
Introducing a black hole into the gravitational side of the AdS/CFT correspondence allows for the calculation of thermodynamic quantities of the field theory using the established black hole thermodynamic relations \cite{Kovtun2005,Son:2007vk,Herzog:2006ra,BallonBayona:2007vp}.
The temperature at which the black hole solution becomes energetically favored corresponds to $T_c$.

To analyze the thermodynamics of an AdS/CFT model, the string frame metric must be changed to an AdS-Schwarzschild metric,
\be
ds^2 = \frac{L^2}{z^2}\left(-f(z) dt^2 +d\vec{x}^2+\frac{dz^2}{f(z)}\right), 
\ee
where $f(z)$ is a function to be determined that sets the location of the horizon of the black hole.
The presence of this metric function will modify the equations of motion (\ref{C}-\ref{G}), and must be solved for using these new background equations.
The potential found in Section \ref{sec:potential} can be used as a starting point for this analysis, and was part of the motivation for constructing such a potential.

With this finite-temperature AdS/CFT model in hand, it will be possible to study a variety of aspects of the deconfined phase of QCD matter, including the entropy, speed of sound, and free energy.
In addition, similar models have been used to study other phenomena relevant to heavy ion collisions, such as jet quenching.
There have been other AdS/CFT models that analyze the finite temperature behavior, but these models suffer from inconsistencies with the zero-temperature meson spectra that the model in Chapter \ref{ch:dynamical_threefield} was designed to address.
A model that correctly describes the meson spectra and zero-temperature chiral symmetry breaking while also allowing for the investigation of finite-temperature effects would be an improvement upon existing work in the field.

In the research described in this thesis, we extended and improved upon the relatively simple AdS/QCD soft-wall model originally proposed in \cite{karch-katz-son-adsqcd} and modified in \cite{gherghetta-kelley}.
We constructed a dynamical model with good phenomenological results for the mass spectra of mesons that is now ripe for extension into finite temperature.
 AdS/QCD is not yet a precision tool for the study of QCD, and a dual model that captures the full complexity of strongly-coupled QCD will require much further work and will likely necessitate the inclusion of stringy effects.
However, we have found that a relatively simple model can give rich results for strongly coupled gauge theories. 




