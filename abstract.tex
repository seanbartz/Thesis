%%%%%%%%%%%%%%%%%%%%%%%%%%%%%%%%%%%%%%%%%%%%%%%%%%%%%%%%%%%%%%%%%%%%%%%%%%%%%%%%
% abstract.tex: Abstract
%%%%%%%%%%%%%%%%%%%%%%%%%%%%%%%%%%%%%%%%%%%%%%%%%%%%%%%%%%%%%%%%%%%%%%%%%%%%%%%%

%%%%%%%%%%%%%%%%%%%%%%%%%%%%%%%%%%%%%%%%%%%%%%%%%%%%%%%%%%%%%%%%%%%%%%%%%%%%%%%%

Gauge/gravity dualities are a tool that allow the analytic analysis of strongly-coupled gauge theories.
The Anti-de Sitter Space/Conformal Field Theory conjecture posits a duality between ten-dimensional string theory and a super Yang-Mills theory.
A phenomenologically-motivated modification of this correspondence is known as AdS/QCD, a duality between strongly-coupled QCD-like theories and weakly-coupled gravitational theories in an additional dimension.
QCD is not scale-invariant, so the dual theory must be modified in the conformal dimension to reflect this. 

This thesis examines ``soft-wall" models of AdS/QCD, wherein the conformal symmetry is broken by a field known as a dilaton.
The dynamics of the dilaton and other background fields are examined, and a potential for these fields is determined.
The background fields are numerically derived from this potential and used in the calculation of meson spectra, which match well to experiment.

The research presented in this thesis is based upon previously-published work contributed to by the author \cite{bartz-pions, Bartz2014}, as well as work that has yet to be published.