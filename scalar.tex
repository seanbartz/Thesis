\chapter{Scalar and Glueball Sectors}
\label{ch:scalar}

The terms of the action relevant for the scalar meson and glueball fields have the form 
\be
\cS =-\frac{1}{16\pi G_5} \int d^5x \sqrt{-g} e^{-2\Phi}  \mathrm{Tr} \left[ |DX|^2  +\partial_M \mathcal{G} \partial^M \mathcal{G} +V_m(\Phi,X^2, \mathcal{G}) \right].
\ee
Here, we are using the exponential representation for the scalar field,
\be
X=\left(S(x,z)+\frac{\chi(z)}{2}\right)e^{2i\pi(x,z)},
\ee
where $S(x,z)$ is the scalar field fluctuations and $\chi(z)$ is the chiral condensate.
The glueball field $\mathcal{G}$ also has a condensate, written as
\be
\mathcal{G} = \left(H(x,z) +\frac{G(z)}{2}\right),
\ee
where $H(x,z)$ are the fluctuations of the glueball field and $G(z)$ is the glueball condensate.

The potential $V_m$ is defined in such a way that $\langle \mathrm{Tr} V_m \rangle= V $, where $V$ is the potential for the background fields in the string frame,
\be
V=-12+4\rt6 \phi + a_0 \phi^2 +\frac{m_X^2}{2} \chi^2 +U +\Delta U,
\ee
where $\phi$ is the re-scaled version of the dilaton, $\phi =\sqrt{8/3} \Phi$.
Here, $U$ is more than quadratic in the fields,
\be
U=a_1 \phi \chi^2 +a_2 \phi G^2 + a_3 \chi^4 + a_4 G^4 + a_5 \chi^2 G^2 +a_6 G^2 \tanh(\phi),
\ee
and $\Delta U$ is assumed to be a function of $\phi$ only.

The terms of $\mathrm{Tr} V_m$  that are relevant for the analysis of the scalar sector are those terms involving $X,\mathcal{G}$,
\be
\mathrm{Tr} V_m \sim m_X^2 X^2 + 2 a_1 \phi X^2 + 2 a_2 \phi \mathcal{G}^2 +8 a_3 X^4 +8 a_4 \mathcal{G}^4 + 8 a_5 X^2 \mathcal{G}^2 +2 a_6\mathcal{G}^2 \tanh{g\phi}.
\ee
Expanding and keeping only terms that are quadratic in the fields $S, H$, we obtain
\be
\mathrm{Tr} V_m \sim \left(m_X^2 + 2 a_1 \phi + 12  a_3 \chi^2 +2 a_5 G^2\right)S^2 + \left(2a_2 \phi +12 a_4 G^2 + 2a_5 \chi^2 + 2 \tanh(g\phi)\right)H^2 + 4 a_5 G\chi H S.
\ee
We can simplify the expression by writing
\be
\mathrm{Tr} V_m \sim \left(m_X^2 +\frac{\partial^2 U}{\partial \chi^2}\right)S^2 + \frac{\partial^2 U}{\partial G^2} H^2 +4a_5 G \chi H S,
\ee
where
\ba
\frac{\partial^2 U}{\partial \chi^2} &=& 2a_1 \phi \chi + 4 a_3 \chi^3 +2a_5 \chi G^2\\
\frac{\partial^2 U}{\partial G^2} &=& 2a_2 \phi G +4 a_4 G^3 +2a_5 \chi^2 G+ 2a_6 G \tanh(g\phi)
\ea


The equations of motion are as follows.
Varying with respect to $S$ gives the equation of motion
\be 
\partial_z(z^{-3}e^{-2\Phi}S')-z^{-5}e^{-2\Phi}\left(m_X^2 + \frac{\partial^2 U}{\partial \chi^2}\right)S -4a_5 z^5 e^{-2\Phi} G\chi H = -z^3 e^{-3} m_S^2 S.
\ee
Varying with respect to $H$ yields the equation of motion
\be
\partial_z(z^{-3}e^{-2\Phi}H')-z^{-5}e^{-2\Phi} \frac{\partial^2 U}{\partial G^2}H -4a_5 z^5 e^{-2\Phi} G\chi S = -z^3 e^{-3} m_S^2 H.
\ee

To put the equations in Schr{\"o}dinger form, we make the substitutions 
\ba
S&\rightarrow&e^{\omega_s/2}S \\ 
H&\rightarrow&e^{\omega_s/2}H,
\ea
where $\omega_s=2\Phi+3\log z$.
This reduces the equations of motion to
\ba
-S''+\left(\oneqt \omega_s'^2-\thalf \omega_s''+m_X^2 -\frac{\partial^2 U}{\partial\chi^2}\right)S +4a_5 G\chi H &=& m_S^2 S \\
-H'' +\left(\oneqt \omega_s'^2-\thalf \omega_s'' -\frac{\partial^2 U}{\partial G^2}\right)H +4a_5 G\chi S &=& m_H^2 H
\ea

These equations need to be solved simultaneously for the eigenvalues $m_S^2$, $m_H^2$.
