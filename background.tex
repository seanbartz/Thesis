\chapter{Background}
\label{background}

In this chapter, we cover the background of gauge/gravity dualities and the AdS/CFT correspondence.
Our focus is on the phenomenological applications of these models, so many of the technical details of the correspondence will not be presented here.
A variety of review articles cover the topic in more mathematical detail

Once we have established the basics of AdS/CFT, we will review the applications of the  duality to hadronic physics known as AdS/QCD.
We will present the simplest AdS/QCD models and the results derived from them, and improve on these results in later chapters.

\section{What is AdS/CFT?}
The general principle of dualities is to describe the same physical systems with two different but equivalent theories. 
The goal of this approach is to find that one of the theories is more analytically tractable than the other, and therefore more useful in describing physical phenomena.
The original work on the AdS/CFT correspondence began with a duality between ten-dimensional string theory on $AdS_5 \times S^5$ and $\mathcal{N}=4 $ Super Yang-Mills Theory.
The details of the string theory are not crucial to the understanding of the derivation of the correspondence.
The constituents of the theory are extended objects known as $p$-branes, where $p$ refers to the spatial dimension of the object. 
One-dimensional $p$-branes are more commonly known as strings, which can be either closed or open.
Closed strings are propagate freely through space, while open strings have  their endpoints on $p$-dimensional Dirichlet-branes (D-branes). 
These strings are characterized by their length $l_s$ and the coupling constant $g_s$ that describes the strength of their interactions. 
These parameters are are related to the $D$-dimensional Newton's constant
\be
G_D \sim g_s^2 l_s^{D-2}.
\ee
The string length can also be related to the Regge slope parameter
\be
\alpha' = l_s^2.
\label{eq:ReggeSlope}
\ee

We motivate the AdS/CFT correspondence by examining a stack of $N$ D3 branes as a background for a type IIB string theory.
We consider the dynamics of this system by taking the limits of low energy and strong coupling. 
The order in which these limits are taken determines the appearance of the result, yielding the two sides of the duality.

Taking the low-energy limit first gives an $SU(N)$ gauge theory, $\mathcal{N} =4 $ Super Yang-Mills (SYM) Theory. 
\footnote{N.B.  $N$ refers to the number of D3 branes and the number of colors in the gauge theory, while $\mathcal{N}$ indicates the number of supercharges in the SYM theory.}
The gauge coupling in this theory is related to the string coupling by
\be
g_{YM}^2=4\pi g_s.
\label{eq:gYM}
\ee
Assuming the string length is small, any interactions involving the open strings in the bulk (the region away from the branes) can be ignored.
We are now free to make the assumption of strong coupling in the SYM theory.
There are now two decoupled components of the low-energy system: the string theory governing the open strings in the bulk, and the strongly-coupled SYM theory on the branes.

Now, let us take the strong coupling limit first. 
In this instance, we have a large number of coincident D-branes resulting in a large energy density, and we must use general relativity to take into account its effect on the curvature of spacetime.
The classical metric that solves the supergravity equations is
\be
 \label{equSGmetric}
ds^{2} = \frac{1}{\sqrt{1+\frac{L^{4}}{r^{4}}}}\left( -dt^{2} + d\vec{x}^{2}\right) + \sqrt{1+\frac{L^{4}}{r^{4}}}(dr^{2}+r^2 d\Omega_5),
\ee
where the curvature radius $L$ sets the scale and is given by\cite{FILL-IN}
\be
L^{4} = 4 \pi g_{s}N \alpha'^{2}.
\label{eq:L4}
\ee
The vector $\vec{x}$ runs over three spatial dimensions, and $d\Omega_5$ is the five-dimensional angular element.

The stack of $N$ D3 branes is located at $r=0$, and to an observer located at $r=\infty$, any excitations near the branes appears to be low energy due to gravitational red-shifting.
As a consequence, taking the low-energy limit is equivalent to taking the limit $r \ll L$.
The metric becomes
\be
\label{equ10drmetric}
ds^{2} = \frac{r^{2}}{L^{2}} (-dt^{2}+d\vec{x}^{2}) + \frac{L^{2}}{r^{2}}dr^{2} + L^2 d\Omega_{5},
\ee
where $r$ is strictly positive.
This metric is a product of five dimensional anti-de Sitter space with a five-dimensional sphere ($AdS_5\times S_5$).
We can re-write this metric using the coordinate transformation
\be
z = \frac{L^{2}}{r}.
\ee
This transformation gives a metric with a single warp factor,
\be
\label{equ5dzmetric}
ds^{2} = \frac{L^2}{z^2}\left(-dt^{2} + d\vec{x}^{2} + dz^{2}\right),
\ee
where  $z>0$, with a UV cut-off at an infinitesimal $z$-value, $z=z_0$.  
The closed string excitations in the bulk are decoupled from the excitations near the branes.

In each of these descriptions, we began with a stack of $N$ D3 branes and ended with a low-energy, strongly-coupled system.
Taking the low-energy limit first results in an $\mathcal{N}=4$ SYM gauge theory, while taking the strong coupling limit first yields a system of excitations on an $AdS_5 \times S_5$ metric.
Maldacena's conjecture is that these two systems should describe the same physics.
Each system has closed strings in the bulk that are decoupled from the rest of the system, so the conjecture results in a proposed duality between the $\mathcal{N}=4$ SYM theory and type IIB string theory in anti-de Sitter space.

This duality is particularly useful if we can ignore all stringy effects and treat the type IIB string theory as a classical supergravity theory on the metric (\ref{equ10drmetric}). 
This approximation is valid if the string length is much less than the radius of curvature,
\be
l_s \ll L.
\ee
We can relate these quantities to the Yang-Mills coupling using (\ref{eq:ReggeSlope}), (\ref{eq:gYM}), and (\ref{eq:L4}), yielding
\be
\frac{L^4}{l_s^4} = g_{YM}^2 N.
\ee
The quantity $g_{YM} N$ is known as the 't Hooft coupling, $\lambda$.
Thus, the requirement that the string length be small is equivalent to requiring the 't Hooft coupling to be large, $\lambda \gg 1$.
In other words, the classical approximation is valid when the dual gauge theory is strongly coupled.
The 't Hooft coupling acts as the effective gauge coupling, which is often expressed as a scalar field called the dilaton, $\Phi$, where
\be
\Phi = \log \lambda.
\ee
The behavior of the dilaton field will take a central role in this thesis.

In summary, Maldacena conjectured that a ten-dimensional string theory is dual to $\mathcal{N} = 4 $ super Yang-Mills theory.
We showed that the regime in which stringy effects can be neglected is equivalent to the strong-coupling limit of the gauge theory.
The usefulness of this duality is evident, because the strong-coupling regime of a gauge theory is the limit in which it is difficult to perform calculations.
Conveniently, this regime is dual to the string theory regime where calculations are easy because they can be done classically.

For our purposes, we reduce the 10-dimensional metric (\ref{equ10drmetric}) to a five-dimensional space by ignoring the $S_5$ manifold and keeping only the $AdS_5$ metric (\ref{equ5dzmetric}).
It turns out that discarding $S_5$ removes the supersymmetry from the SYM theory, resulting in a non-supersymmetric conformal field theory.
In the end, we have a duality between a classical gravity theory in five dimensional anti-de Sitter space and a four-dimensional conformal field theory, our AdS/CFT correspondence.
It is this relationship between a field theory in four dimensions and a gravitational theory in five dimensions that lends these models the evocative name \it{holography}.

