\chapter{Gell-Mann--Oakes--Renner Derivation}
\label{app_GMOR}

 A quark-antiquark pair may spontaneously appear from the vacuum, in keeping with Heisenberg's Uncertainty Principle. 
 This pair will have zero total momentum and angular momentum, with a left-handed quark paired with a right-handed antiquark, and vice versa, resulting in a nonzero net chirality.
 The QCD vacuum contains a condensate of these chiral quark-antiquark states, resulting in a nonzero vacuum expectation value for the chiral operator
 \be
 \langle\bar{q}q\rangle = \langle \bar{q}_L q_R +\bar{q}_R q_L \rangle \neq 0.
 \ee
 Thus, the vacuum mixes the two quark helicities, causing the light quarks to acquire an effective mass as they move through and interact with the vacuum.
 
 We can explore the relationship of spontaneous chiral symmetry breaking to the pion by parameterizing the matrix element of the axial current
  \be
  j^{\mu a}_5(x)=\bar{q}\gamma^\mu\gamma_5 \tau^a q
  \ee
   between the vacuum and a pion \cite{Peskin:qft}
 \be
 \langle0|j^{\mu a}_5(x)|\pi^b(p)\rangle = -i p^\mu f_\pi \delta^{ab} \EXP^{-ip\cdot x},
 \label{eq:pidecay}
 \ee
 where $f_\pi$ is known as the \emph{pion decay constant} and has dimensions of mass.
 Taking the derivative $\partial_\mu$ of the left-hand side of (\ref{eq:pidecay}) is equivalent to contracting the right-hand side with $p_\mu$, yielding
\be
\langle0|\partial_\mu j^{\mu a}_5(x)|\pi^b(p)\rangle = -p_\mu(i p^\mu f_\pi \delta^{ab}\EXP^{-ip\cdot x}) = -ip^2  f_\pi \delta^{ab}\EXP^{-ip\cdot x}.
\label{eq:axialCurrent}
\ee
Taking $x=0$, and making the substitution $p^2 = m_\pi^2 $ for an on-shell pion, we obtain the so-called partially conserved axial current (PCAC) relation
\be
\partial_\mu j^{\mu a}_5 = f_\pi m_\pi^2 \pi^a.
\label{eq:PCAC}
\ee
In the limit of zero quark mass, the axial current is exactly conserved, i.e. $\partial_\mu j^{\mu5}(x)=0$, implying that $m_\pi^2=0$, and the pion is massless, as required by Goldstone's theorem.
  
When the quark masses are not equal to zero, the chiral symmetry is said to be broken explicitly. 
This is because the QCD Lagrangian contains terms of the form $  m_q \bar{q}q = m_q (\bar{q}_L q_R + \bar{q}_R q_L)$, which requires a mixing of the chiral components.
The axial field is no longer exactly conserved, becoming
\be
\partial_\mu j^{\mu a}_5(x) = i\bar{q}\{M,\tau^a\}\gamma_5 q,
\ee
where M is the mass matrix
\be
M=\left( \begin{array} {cc}
m_u&0 \\
0 &m_d 
\end{array}\right) \,.
\ee
Using this equation with \ref{eq:axialCurrent}, we find
\be
 \langle0|\partial_\mu j^{\mu a}_5(0)|\pi^v(p)\rangle = -p^2 f_\pi \delta^{ab} = \langle 0| i \bar{q}\{M,\tau^a\}\gamma_5 q | \pi^b(p)\rangle.
 \ee
 
 We derive the Gell-Mann--Oakes--Renner relation using the argument presented in \cite{Greiner}.
 We calculate the quantity
 \be
 I=\int d^4 x e^{-ip\cdot x} \left \langle 0 \left | \mathrm{Tr} \left(\partial^\mu j^{a}_{5\mu}(0) \partial^\nu j^{b}_{5\nu}(x)\right)\right |0\right\rangle.
 \ee
 Using the PCAC relation (\ref{eq:PCAC}), we can substitute the pion operators and reduce the expression to the pion propagator,
 \ba
 I &=& f_\pi^2 m_pi^4 \int d^4x e^{-ip\cdot x} \left \langle 0 \left | \mathrm{Tr} \left(\pi^a(0) \pi^b(x) \right)\right |0\right\rangle \\
 &=& f_\pi^2 m_pi^4 \delta^{ab} i D_\pi(p).
 \ea
 Taking the limit $p\rightarrow0$, the propagator becomes $D_\pi=-1/m_\pi^2$, and the expression for $I$ reduces to 
 \be
 I = -i f_\pi^2 m_\pi^2.
 \label{eq:I1}
 \ee
 Alternatively, we can calculate the quantity $I$ by integrating by parts, 
 \ba
 I&=&\int d^4x \left(-\partial^\nu e^{-ip\cdot x} \right) \left \langle 0 \left | \mathrm{Tr} \left(\partial^\mu j_{5\mu}^{a}(0) j_{5\nu} ^{b}(x)\right)\right | 0 \right \rangle \nonumber \\
 &-&\int d^4x \, e^{-ip\cdot x} \big \langle 0 \big | \big ( \partial^\nu \Theta(t) j_{5\nu}^{b}(x)\partial^\mu j_{5\mu}^{a} (0) \nonumber \\
 &+& \partial^\nu \Theta(-t) \partial^\mu j_{5\mu}^{a}(0) j_{5\nu}^{b} (x) \big) \big | 0 \big \rangle,
 \ea
 where $\Theta$ is the Heaviside step function.
 In the limit $p\rightarrow 0$, the first term vanishes.
 We can write the second and third terms as a commutator, and the Heaviside function becomes a Dirac delta,
 \be
 I = \int d^4 x \, \delta(t) \left \langle0\left | \left[ j^{b}_{5t}(x),\partial^\mu j_\mu^{5a}\right ] \right | 0  \right\rangle.
 \ee
 Integrating over the delta function, we obtain
 \be
 I=\int d^3x \, \left \langle 0 \left | \left [ j_{5t}^{b}(0,\vec{x}),\partial^t j_{5t}^{a}(0,0) \right ]  \right | 0 \right \rangle.
 \ee
 
 Because the axial current $j_{5t}^{a}$ is not explicitly time-dependent, the time derivative can be re-written as a commutator with the Hamiltonian,
 \be
 I = i \int d^3x \, \left \langle 0 \left | \left [ j_{5t}^{b}(0,\vec{x}), [j_{5t}^{a}(0,0),H(0) ] \right] \right | 0 \right \rangle .
 \ee
Using the mass term from the Hamiltonian,
\be
H(0) = m_u \bar{u}(0,\vec{x})u(0,\vec{x}) + m_d \bar{d}(0,\vec{x})d(0,\vec{x}),
\ee
and the commutation relation for the quark states
\be
[q_a^\dagger(\vec{x},t),q_b(\vec{y},t)] = \delta_{ab} \delta^3(\vec{x}-\vec{y}),
\ee
the quantity $I$ becomes
 \be
 I = (m_u+m_d) \left \langle 0 \left | (\bar{u}u +\bar{d}d) \right | 0 \right \rangle.
 \label{eq:I2}
 \ee
 The quark condensate is defined $\sigma = \left \langle 0 \left | (\bar{u}u +\bar{d}d) \right | 0 \right \rangle$.
 In this thesis, we assume that $m_u=m_d \equiv m_q$.
 We compare the two expressions for $I$,\ref{eq:I1} and \ref{eq:I2}, finding the Gell-Mann--Oakes--Renner relation to be
 \be
 2 m_q \sigma = f_\pi^2 m_\pi^2.
 \ee
