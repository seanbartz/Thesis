\chapter{Numerical Methods for solving Ordinary Differential Equations}
\label{appendix_numerical}

Ordinary differential equations can always be reduced to a system of coupled first-order differential equations, which is advantageous for numerical solutions.
For example, the second-order equation
\be
y''(x)+f(x)y'(z) = g(x)
\ee
can be rewritten as 
\ba
y'(x) &=& z(x) \\
z'(x) &=& g(z) -f(x)z(x)
\ea

Written generically, an ordinary differential equations system is reduced to a set of $N$ first-order differential equations
\be
y_i'(x)=f_i(x,y1,\dots,y_N) \quad \quad i=1,\dots, N
\label{eq:ODE}
\ee
where the functions $f_i$ are some known set of equations involving the coordinate $x$ and the dependent variables $y_i$.
For a unique solution, one must specify a set of $N$ boundary conditions.
If these conditions are all specified at a single point, $x_0$, this is known as an initial value problem, and one can simply choose an integration method and calculate the values for $f_i$ over the desired domain.
When some of the conditions are specified at point $x_0$ and the rest are specified at the point $x_f$, this is known as a boundary value problem, and one must use one of the methods detailed below.

\section{Shooting Method}
The shooting method turns boundary value problems into initial value problems.
The classic example of a shooting method can be visualized as the firing of a cannon, with boundary conditions given by the location of the cannon and the location of the target. 
This is changed to an initial value problem by selecting an arbitrary value for the angle of the cannon and firing. 
The angle of the cannon is incremented until the cannonball hits the target, which matches the final boundary condition.

More concretely, the boundary value problem is written as in (\ref{eq:ODE}), with boundary conditions
\ba
B_{1j}(x_1,y_1,\dots, y_N) &=& 0 \quad \quad j=1,\dots, n_1 \\
B_{2k}(x_2,y_1,\dots, y_N) &=& 0 \quad \quad k=1,\dots, n_2 
\ea
with $n_1$ boundary conditions at point $x_1$ and $n_2=N-n_1$ at point $x_2$.
These boundary conditions generically can be any algebraic combination of the variables, but in this work they are always of the form $y_i(x) = 0$.

To convert this to an initial value problem, set the $n_1$ components of $\bf{y}$ to the values specified by the boundary conditions at $x_1$, and set the other components of $\bf{y}$ to arbitrary values.
Once this initial value is solved, the values at $x_2$ are compared to the boundary conditions $B_2$.
This process is repeated with an altered set of arbitrary values for $\bf{y}$. 
A root-finding routine is used to find the values that satisfy the boundary conditions.

A similar method can be used for eigenvalue problems, sets of differential equations that have a solution only for particular values of a certain parameter.
In such problems, the eigenvalue is of greater interest than the behavior of the function itself. 
Because of this, the overall normalization of the solution is unimportant, and the initial values can remain arbitrary. 

\begin{figure}[htb]
\center{\includegraphics[width=300pt]{shooting.pdf}}
\caption{An illustration of the shooting method.}
\label{fig:shooting}
\end{figure}