\chapter{Numerical Methods for solving Ordinary Differential Equations}
\label{appendix_numerical}

Ordinary differential equations can always be reduced to a system of coupled first-order differential equations, which is advantageous for numerical solutions.
For example, the second-order equation
\be
y''(x)+f(x)y'(z) = g(x)
\ee
can be rewritten as 
\ba
y'(x) &=& z(x) \\
z'(x) &=& g(z) -f(x)z(x)
\ea

Written generically, an ordinary differential equations system is reduced to a set of $N$ first-order differential equations
\be
y_i'(x)=f_i(x,y1,\dots,y_N) \quad \quad i=1,\dots, N
\label{eq:ODE}
\ee
where the functions $f_i$ are some known set of equations involving the coordinate $x$ and the dependent variables $y_i$.
For a unique solution, one must specify a set of $N$ boundary conditions.
If these conditions are all specified at a single point, $x_0$, this is known as an initial value problem, and one can simply choose an integration method and calculate the values for $f_i$ over the desired domain.
When some of the conditions are specified at point $x_0$ and the rest are specified at the point $x_f$, this is known as a boundary value problem, and one must use one of the methods detailed below.

\section{Shooting Method}
The shooting method turns boundary value problems into initial value problems.
The classic example of a shooting method can be visualized as the firing of a cannon, with boundary conditions given by the location of the cannon and the location of the target. 
This is changed to an initial value problem by selecting an arbitrary value for the angle of the cannon and firing. 
The angle of the cannon is incremented until the cannonball hits the target, which matches the final boundary condition.

More concretely, the boundary value problem is written as in (\ref{eq:ODE}), with boundary conditions
\ba
B_{1j}(x_0,y_1,\dots, y_N) &=& 0 \quad \quad j=1,\dots, n_1 \\
B_{2k}(x_f,y_1,\dots, y_N) &=& 0 \quad \quad k=1,\dots, n_2 
\ea
with $n_1$ boundary conditions at point $x_0$ and $n_2=N-n_1$ at point $x_f$.
These boundary conditions generically can be any algebraic combination of the variables.

For concreteness, consider a second-order eigenvalue problem of the type considered in this thesis,
\be
\psi_n''(z) + V(z) \psi_n(z) = m_n^2 \psi_n(z).
\ee
This can be reduced to two first-order differential equations by making the substitutions $y_1=\psi$, $y_2=\psi'$, resulting in the system
\ba
y_1' &=& y_2 \\
y_2' &=& \left(m_n^2 -V(z) \right)y_1.
\ea
The boundary conditions are $y_1(z_0) = y_1(z_f) = 0$. 
\footnote{In the soft-wall model, we should take the limit $z_f\rightarrow \infty$.
This is impossible to do numerically, but it is easy to make $z_f$ sufficiently large that the eigenvalues are not effected.}
In this instance, the eigenvalues are of greater importance than the behavior of the solutions $y_1, y_2$, so the overall normalization of the solutions is arbitrary.

This boundary value problem is changed to an initial value problem by setting 
\ba
y_1(z_0) = 0 \\
y_2(z_0) = c,
\ea
where $c$ is an arbitrary constant. 
A small test value for $m_n$ is chosen, and the initial value problem is solved. 

\begin{figure}[htb]
\center{\includegraphics[width=300pt]{shooting.pdf}}
\caption{An illustration of the shooting method.}
\label{fig:shooting}
\end{figure}

As illustrated in Figure \ref{fig:shooting}, when the test value for $m_n$ is less than the first eigenvalue $\lambda_1$, the value of $y_1(x_f)$ is positive.
As $m_n$ is incremented, eventually $y_1(x_f)$ becomes negative.
A root-finding routine is used to approximate the values $\lambda_n$ such that $y_1(x_f)=0$. 
The number of antinodes in the wavefunction indicates to which excitation mode the eigenvalue corresponds.

\section{Matrix Method}\label{appMatrix}

The coupled second-order 
The equations of motion can be reduced to a set of second-order differential equations,
\begin{eqnarray}
-\varphi'' + V_1(z)  \varphi + f(z)\pi  = 0, \label{equphiAppend}\\
-\pi'' + V_2(z) \pi + g(z)\varphi = 0, \label{equpiAppend}
\end{eqnarray}
where the eigenvalues are contained within the coefficient functions. 
These equations can be expressed as a system of first-order differential equations
\begin{equation} \label{equmatrixphi}
\Phi' + W(z) \Phi = 0,
\end{equation}
where $W$ is the matrix
\begin{equation}
W = \left(\begin{array}{cccc}
0     & 1     & 0     & 0 \\ 
V_1(z)& 0     &   f(z)& 0 \\
0     & 0     & 0     & 1 \\
g(z)  & 0     & V_2(z)& 0
\end{array}\right)
\end{equation}
and $\Phi$ is the vector
\begin{equation}
\Phi_{\alpha i} = \left(\begin{array}{c} \varphi_{i} \\
			-\varphi'_{i} \\
			\pi_{i} \\
			-\pi'_{i} \end{array}\right)		
\end{equation}
that forms an orthonormal basis of solutions. 
We can propagate the solution $\Phi$ between two boundary points
\begin{equation}
\Phi(z_f) = U(z, z_f, z_0, m_n^2)\Phi(z_0),
\end{equation}
where we solve (\ref{equmatrixphi}) with the appropriate boundary condition at $z_0$. 
The eigenvectors and eigenvalues of $U$ are then calculated. 
Two of the solutions are divergent, but the solutions that correspond to the two smaller eigenvalues do not diverge.
We label the non-divergent solutions as the eigenvectors $u_3$ and $u_4$. 
The non-divergent solutions for $\Phi_{i}$ can be written as a linear combination of $u_3$ and $u_4$
\begin{equation}
\Phi_{i} = \alpha u_3 + \beta u_4.
\end{equation}   
Non-trivial values of  $\alpha$ and $\beta$ satisfy
\begin{equation}\label{equNontrivial1}
\left(\begin{array}{cc} u_3^1 & u_4^1 \\ u_3^{3} & u_4^{3} \end{array}\right)\left(\begin{array}{c} \alpha \\ \beta\end{array}\right) = 0
\end{equation}
for Dirichlet or
\begin{equation}\label{equNontrivial2}
\left(\begin{array}{cc} u_3^2 & u_4^2 \\ u_3^{4} & u_4^{4} \end{array}\right)\left(\begin{array}{c} \alpha \\ \beta\end{array}\right) = 0
\end{equation}
for Neumann boundary conditions. 
The values of $m_n^2$ that minimize the determinant of the matrix (\ref{equNontrivial1}) or (\ref{equNontrivial2}) are the eigenvalues of the system (\ref{equphiAppend}-\ref{equpiAppend}). 
An abrupt change in its behavior of the plot of the determinant vs. the value of $m_n$ signals the location of an eigenvalue. 